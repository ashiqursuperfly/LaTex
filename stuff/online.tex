\documentclass[14pt]{article}
\usepackage{verbatim}
\usepackage{graphicx}
\usepackage{subcaption}
\usepackage{amsmath}

\title{CSE 300 Online 2}
\author{1605106}
\date{\today}


\begin{document}
	\maketitle
	\listoffigures
	\tableofcontents
	\newpage
	\section{Introduction}
	Google Scholar is a wonderful search engine for finding research articles. It
	is freely accessible \and indexes the full literature or metadata of scholarly articles across various formats and disciplines \cite{wiki_gs}.
	
	\begin{figure}[h]
		\centering
		\begin{subfigure}{0.15\textwidth}
			\centering
			\includegraphics[scale=0.1]{gs_icon.png}
			\caption{Icon}
		\end{subfigure}
		~
		\begin{subfigure}{0.15\textwidth}
			\centering
			\includegraphics[scale=0.1]{gs_page.png}
			\caption{Banner}
		\end{subfigure}
		\\
		\begin{subfigure}{0.4\textwidth}
			\centering
			\includegraphics[scale=0.5]{plag.jpg}
			\caption{Say no to plagiarism}
		\end{subfigure}
	\caption{Guidance towards Research}
	\end{figure}

	\section{Equations}
	Euler’s formula is one of the most important equations in mathematics. It estab-
	lishes a relationship between trigonometric function and complex exponential
	function. The equation is as follows
	
	\begin{equation}
		e^{i \theta} = \sin\theta + \imath \cos\theta
	\end{equation}
	If we put $ \theta = \frac{\pi}{2}$ in equation 1, we get the following
	\begin{align*}
		e^{\imath\frac{\pi}{2}} & = 
		\cos\frac{\pi}{2} + \imath \sin\frac{\pi}{2}\\
		& = 0 + \imath.1\\
		& = \imath
	\end{align*}
	If we put $\theta = \pi $, we get $e^{i\pi} + 1 = 0$ which is termed as Euler’s Identity \cite{wiki:euler}.
	\subsection{Equations Samples:}
	
	\begin{equation}
		{{n}\choose{r}}  = \binom{n}{r} =  \frac{n !}{r ! (n-r) !} 		
	\end{equation}
	
	

	
	
	
	
	
	
	

	\bibliographystyle{plain}
	\bibliography{ref_lib}
\end{document}