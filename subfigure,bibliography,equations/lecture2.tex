\documentclass{article}

\usepackage{verbatim}
\usepackage{graphicx}
\usepackage{subcaption}
\usepackage{amsmath}

\title{Lecture 2}
\author{Tareq}

\begin{document}
\maketitle
\tableofcontents
\listoffigures
\newpage


\section{Verbatim}

usepackage command : \verb|\usepackage{verbatim}|.\textbackslash.


\section{Figure}

\begin{figure}[h]
	\centering
	\includegraphics[width=0.3\textwidth]{buetlogo.png}
	\caption{Logo of BUET}
	\label{fig:buetlogo}
\end{figure}


\begin{figure}[h]
	\centering
	\begin{subfigure}{0.3\textwidth}
		\includegraphics[width=0.8\textwidth]{ubuntulogo.png}
		\caption{Ubuntu}
	\end{subfigure}
	~
	\begin{subfigure}{0.3\textwidth}
		\includegraphics[width=0.8\textwidth]{kalilogo.png}
		\caption{Kali Linux}
	\end{subfigure}
	~
	\begin{subfigure}{0.3\textwidth}
		\includegraphics[width=0.8\textwidth]{ubuntulogo.png}
		\caption{Ubuntu}
	\end{subfigure}
	
	\caption{Linux distributions}
\end{figure}


\section{Bibliography}

This book \cite{deng2009imagenet} is amazing. 
This book \cite{knuth1989mathematical} is good too

\section{Mathematical Equation}

%text a + a = 2a\\
Hello world as u can see this is an inline equation.$ test-a + a = 2a $ Lets keep this as it is and leave it be.\\Hello world as u can see this is a specialized equation.$$ test-a + a = 2a $$ Lets keep this as it is and leave it be.

\subsection{Subscript and SuperScript}
$$ a^2 \quad a_22 \quad a_{in} \quad a_{2} $$

\subsection{Methods for writing an Equation}
\textbf{NOTE}\\
1.wrap equation with \textdollar \textdollar \space for specialized Equation\\
2.wrap equation with \textdollar \space for inline Equation\\
3. Wrap with \verb|\begin{equation}| ... \verb|\end{equation}| for generating an automated id number for the equation. We can use \textbackslash label and \textbackslash ref to later cross ref this equation number. 



\subsection{Math Functions}

$$ \int^a_b f(x)dx $$

$$ \frac{\tau}{\pi} $$


\begin{equation}
	\begin{bmatrix}
		1 & 0 & 0 \\
		0 & 1 & 0 \\
		0 & 0 & 1 \\
	\end{bmatrix}
\end{equation}


\begin{equation}
\begin{matrix}
1 & 0 & 0 \\
0 & 1 & 0 \\
0 & 0 & 1 \\
\end{matrix}
\end{equation}


\bibliographystyle{plain}
\bibliography{bibfile} 



\end{document}
