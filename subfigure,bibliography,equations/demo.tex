\documentclass[a4paper, 12pt]{article}

\title{Subfigure, Math, Bibliography}
\author{Ashiq}
\usepackage{graphicx}
\usepackage{subcaption}
\usepackage{amsmath}
%\usepackage{mathtools}

\begin{document}
	\maketitle
	\tableofcontents
	\listoffigures
	\newpage
	
	\section{Subfigure}
	\begin{figure}[h]
		\begin{subfigure}[t]{0.2\textwidth}
			\centering
			\includegraphics[width=\textwidth]{star.png}
			\caption{Small}
		\end{subfigure}
		~
		\begin{subfigure}[b]{0.6\textwidth}
			\centering
			\includegraphics[width=\textwidth]{star.png}
			\caption{Big}
		\end{subfigure}
		\caption{Beach}	
	\end{figure}
	\section{Math}
	
	\subsection{Inline}
		$\forall x, x$ is positive and  $x \leq 5$.
	\subsection{Equation}
	
	\subsubsection{Algebra}
		\begin{equation}
			x = y
		\end{equation}
		
		\begin{equation*}
			x_1^2 + x_2 = \sqrt{x+y}
		\end{equation*}
		
		\begin{equation}
			\binom{n}{k} = \frac{n!}{k!(n-k)!}
		\end{equation}
				
	\subsubsection{Trigonometric}
		\begin{equation}
			\sin^2\theta + \cos^2\theta = 1
		\end{equation}
		\begin{align*}
			\cos^2\theta & = \frac{1}{2}.2\cos^2\theta\\
			& = \frac{1}{2}(1+\cos2\theta)
		\end{align*}
	
	\subsubsection{Calculus}
		\begin{equation}
			\lim_{x \to 0}	\frac{\sin x}{x} = 1
		\end{equation}
		\begin{equation}
			\frac{d}{dx}e^x = e^x			
		\end{equation}
		
		\begin{equation}
		\frac{\partial}{\partial x}e^x = e^x			
		\end{equation}
		
		\begin{equation}
			\int x^3 dx = \frac{x^4}{4} + c			
		\end{equation}
		
		\begin{equation}
			\int_b^a f(x)dx
		\end{equation}
	\subsubsection{Showing Multiple Lines Of Calculations}
		\begin{align*}
			\cos^2\theta & = 
			\frac{1}{2}.2\cos^2\theta\\
			& = \frac{1}{2}.(1 + \cos 2\theta)
		\end{align*}
		\subsubsection{Piece-Wise Functions}
		\begin{equation}
			F(x) =
			\begin{cases}
			100 \ & \text{if} \ x > 0 \\0 \ & \text{otherwise}
			\end{cases} 
		\end{equation}				
		
	\subsubsection{Miscellaneous}
		\begin{equation}
			\bigcup\limits_{i=1}^{n} A_i \leq \sum_{i=1}^{n} |A_i|
		\end{equation}
		\begin{equation}
			\bigcup_{i=1}^{n} A_i \leq \sum_{i=1}^{n} |A_i|
		\end{equation}
	In algebra, a quadratic equation is any equation having the form $ax^2+bx+c=0$ where $x$ represents an unknown, and $a$, $b$, and $c$ represent known numbers, with
	$a\neq0$. It can easily be seen, by polynomial expansion, that the following
	equation is equivalent to the quadratic equation:
	$${\left(x+\frac{b}{2a}\right)}^2=\frac{b^2-4ac}{4a^2}$$
	
	Taking the square root of both sides, and isolating $x$, gives:
	\begin{equation}
	x = \frac{-b\pm\sqrt{b^2-4ac}}{2a}
	\end{equation}
	
	\subsection{Some Equations:}
	$$f_1(t) = \int_{3}^{5}sin(x)dx$$
	
	$$F(x) = A_0 + \sum_{n=1}^{N}\left[A_ncos\left(\frac{2\pi nx}{P}\right)+B_nsin\left(\frac{2\pi nx}{P}\right)\right]$$
	
	$$\lim\limits_{x\rightarrow a}\frac{f(x)-f(a)}{x - a}$$
	
	$$\binom{a}{b + c} \binom{\frac{n^2-1}{2}}{n+1}$$
	
	$$h\leq \sqrt{\frac{(s-a)(s-b)(s-c)}{s}}$$
	
	$$6CO_2 + 6H_2O \rightarrow C_6H_{12}O_6 + 6O_2$$
	
	$$\frac{1}{\log_2 x}$$
		
		
	\section{Bibliography}
		For any help, take a look at \cite{Wiki}.
	
	\bibliographystyle{plain}
	\bibliography{bibfile}
	
\end{document}